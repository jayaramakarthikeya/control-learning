\documentclass[11pt, reqno, letterpaper, twoside]{amsart}
\linespread{1.2}
\usepackage[margin=1.25in]{geometry}

\usepackage{amssymb, bm, mathtools}
\usepackage[usenames,dvipsnames,svgnames,table]{xcolor}
\usepackage[pdftex, xetex]{graphicx}
\usepackage{enumerate, setspace}
\usepackage{float, colortbl, tabularx, longtable, multirow, subcaption, environ, wrapfig, textcomp, booktabs}
\usepackage{pgf, tikz, framed}
\usepackage[normalem]{ulem}
\usetikzlibrary{arrows,positioning,automata,shadows,fit,shapes}
\usepackage[english]{babel}
\newtheorem{definition}{Definition}

\usepackage{microtype}
\microtypecontext{spacing=nonfrench}

\theoremstyle{plain}
\newtheorem{theorem}{Theorem}

\theoremstyle{definition}
\newtheorem{solution}[theorem]{Solution}

\usepackage{times}


\begin{document}

\begin{definition}
    Let \textbf{F} be space spanned by observables $ \Phi: \mathbb{R}^n \rightarrow \mathbb{R}^N $ where $N > n$ then the Koopman operator
    $\textbf{K}: \textbf{F} \rightarrow \textbf{F}$ is a infinite-dimensional linear operator (composition operator) which acts on $\Phi$ which spans $\textbf{F}$ such that:
    \begin{equation}
        \textbf{K}(\Phi(x)) = \Phi \circ f(x)
    \end{equation}

    where $f(x)$ is a trasition map of uncontrolled dynamical system and $\Phi(x)$ can be any lifted function.
\end{definition}

In our case we will use finite approximation of Koopman operator using eigenvalue decomposition.

\begin{definition}
    There are good ways to generalize the Koopman operator to controlled systems. Let's consider a way where $ \chi = \begin{bmatrix}
        x \\
        u
    \end{bmatrix}$, and $ \chi \in \mathbb{R}^n \times l(U)$ is an extended state space with control inputs. Then we can define flow map as: $ f(x,u) : \mathbb{R}^n \times l(U) \rightarrow \mathbb{R}^n $ and observables as $\Phi(\chi): \mathbb{R}^N \times l(U) \rightarrow \mathbb{R}^N $. The Koopman operator on extended subspace takes in the form of:
    similar to previous definition:
    \begin{equation}
        \begin{split}
            \textbf{K}(\Phi(\chi)) &= \Phi \circ f(\chi) \\
            \textbf{K}(\Phi(x,u)) &= \Phi \circ (f(x,u),u)
        \end{split}
    \end{equation}
\end{definition}

In the real senario for desgining dynamics of the robot, we sample the data points in discrete domain. Let's define Koopman operator for discrete time 
with sampling rate of $\Delta t$ and timestep k:
\begin{equation}
    \textbf{K}_{\Delta t} (\Phi(x_k,u_k)) = \Phi(f(x_k,u_k),u_k) = \Phi(x_{k+1},u_k)
\end{equation}

\vspace{15pt}
\textbf{Koopman Eigenvalue Decomposition}

Let's take L observables or mesurements of the system and represent it as:

\begin{equation}
    \Phi(\textbf{x}) = \begin{bmatrix}
        \Phi_1(\textbf{x}) \\
        \Phi_2(\textbf{x}) \\
        . \\
        . \\
        \Phi_l(\textbf{x})
    \end{bmatrix} = \sum_{j=1}^{\infty} \varphi_j(\textbf{x})v_{j}
\end{equation}

where $\varphi_j(\textbf{x})$ is Koopman eigenfunction of given state space and $v_{j}$ are known as Koopman modes.  
Koopman eigenfunctions spans the Hilbert space and the observables which are mesurement functions projected from original state space. The Koopman modes which
are defined points in the Hilbert space for given observables. For consvertive systems, Koopman eigenfunctions are orthonormal to each other.
Thus, by applying Koopman operator to given observables we get:
\begin{equation}
    \textbf{K}\Phi(\textbf{x}) = \sum_{j=1}^{\infty} \lambda_j \varphi_j(\textbf{x})v_{j}
\end{equation} 

where ${(\lambda_j,\varphi_j, v_{j})}$ are known as Koopman eigenvalue decomposition and to get finite approximation we take top p eigenvalues 
where $ \lambda_i \in \sigma(\textbf{K})$ and $ \lambda_1 > ... > \lambda_p $ so we can rewrite as:
\begin{equation}
    \textbf{K}\Phi(\textbf{x}) = \Phi(f(x)) = \sum_{j=1}^{p} \lambda_j \varphi_j(\textbf{x})v_{j}
\end{equation}

Let's take snapshots from the data dynamics of the data from $t = 1, .., n+1$ and represent it as
sparse matrix. But we use lift into observable space so that the dynamics are linear and we can use linear state equation.
\begin{equation}
    \begin{split}
        X &= \begin{bmatrix}
            x_1 & x_2 & . & . & x_n
        \end{bmatrix} \\
        X' &= \begin{bmatrix}
            x_2 & x_3 & . & . & x_{n+1}
        \end{bmatrix} \\
        where \hspace{4pt} z_i \in \mathbb{R}^N, x_i \in \mathbb{R}^n, N >> n 
    \end{split}
\end{equation}
and $\Omega$ is defined as:
\begin{equation}
    \Omega = \begin{bmatrix}
        X \\
        U'
    \end{bmatrix} \in \mathbb{R}^n \times l(U)
\end{equation}

Then we can represent as linear system equation:
\begin{equation}
    \begin{split}
        \Omega' &= \begin{bmatrix}
            A & B 
         \end{bmatrix} \begin{bmatrix}
            X \\ U'
         \end{bmatrix} \\
         &= G \Omega \\
         G &= \arg \min_{G} \left \| \Omega' - G\Omega \right \| = \Omega'\Omega^{+}
    \end{split}
\end{equation}
where $+$ is pesudo inverse of the matrix and $\Omega' = \begin{bmatrix}
    X' \\
    U'
\end{bmatrix} $.

As $\Omega$ is high dimensional matrix which can be approximated using SVD as:
\begin{equation}
    \Omega = \tilde{U}\tilde{\Sigma}\tilde{V}^*
\end{equation}

$\tilde{U}$ matrix must be split into two matrices such as $ \begin{bmatrix}
    \tilde{U_1}^* & \tilde{U_2}^*
\end{bmatrix}^*  $ provides basis for X and U'. Well, $\tilde{U}$ will provide reduced basis for input space, while $\hat{U}$ will provide for
evolved space as:
\begin{equation}
    \Omega' = \hat{U}\hat{\Sigma}\hat{V}^*
\end{equation} 
Then we to find solution G by projecting to evolved space basis as:
\begin{equation}
    \tilde{G} = \hat{U}^*G\begin{bmatrix}
        \hat{U} \\ I
    \end{bmatrix}
\end{equation}
By subsitiuting and solving equations we get:
\begin{equation}
    \begin{split}
        \tilde{A} &= \hat{U}^*\tilde{A}\hat{U} = \hat{U}^*Z'\tilde{V}\tilde{\Sigma}^{-1}\tilde{U}_1^*\hat{U} \\
        \tilde{B} &= \hat{U}^*B = \hat{U}^*Z'\tilde{V}\tilde{\Sigma}^{-1}\tilde{U}_2^*
    \end{split}
\end{equation}

\begin{equation}
    \begin{split}
        RMSE \left(\begin{bmatrix}
            x \\
            \dot{x} \\
            \theta \\
            \dot{\theta}
        \end{bmatrix}\right) &= \begin{bmatrix}
            0.995 \\
            1.0319 \\
            0.278 \\
            0.962
        \end{bmatrix} \\
        RMSE \left(\begin{bmatrix}
            x \\
            \dot{x} \\
            \theta \\
            \dot{\theta}
        \end{bmatrix}\right) &= \begin{bmatrix}
            0.018 \\
            1.237 \\
            0.0923 \\
            0.0196
        \end{bmatrix}
    \end{split}
\end{equation}

\end{document}